\section{混合整数线性规划}
\subsection{描述}
线性规划是一个优化问题,其标准形式如下
\[max\{c^Tx|Ax\leq b,x\geq 0\}\]
其中$A\in\mathbb{R}^{m,n}$,$b\in\mathbb{R}^m$,
$c\in\mathbb{R}^n$,$x\in\mathbb{R}^n$。
若$x$的某些元素只能为整数则称为混合线性规划。
很多优化问题都可以归结为上述标准形式。
例如对于没有非负限制的变量$x_0$,
可令$x_0=u-v$,其中$u,v$具有非负限制即可。

\subsection{Sage实现}
这部分内容主要参考以下链接

\href{http://www.sagemath.org/doc/reference/numerical/sage/numerical/mip.html}{http://www.sagemath.org/doc/reference/numerical/sage/numerical/mip.html}

\href{http://www.sagemath.org/doc/reference/numerical/sage/numerical/optimize.html}{http://www.sagemath.org/doc/reference/numerical/sage/numerical/optimize.html}

\subsubsection{线性规划}
假设有一个线性规划模型,要求目标函数$-4x_1-5x_2$的最小值,约束条件为
\[
\begin{array}{rcl}
2x_1+x_2 &\leq& 3\\
x_1+2x_2 &\leq& 3\\
x_1 &\geq& 0\\
x_2 &\geq& 0
\end{array}
\]
其中$x_1,x_2$均为实数,
用$c$记录目标函数的系数向量,$G$记录约束条件的系数矩阵,
$h$记录约束条件的常数向量,使用如下代码即可求解,其中RDF表示系数为实数
\begin{sageblock}
c=vector(RDF,[-4,-5])
G=matrix(RDF,[[2,1],[1,2],[-1,0],[0,-1]])
h=vector(RDF,[3,3,0,0])
sol=linear_program(c,G,h)
\end{sageblock}
求解结果当$(x_1,x_2)$取值为$\textrm{sol['x']}=\sage{sol['x']}$时,
目标函数取得最小值$\sage{-4*sol['x'][0]-5*sol['x'][1]}$。
如下图,图中红点为最小值点

\sageplot[width=\textwidth]{polygon([(0,0),(0,3/2),(1,1),(3/2,0)])+line([(-1,13/5),(4,-7/5)])+point((sol['x'][0],sol['x'][1]),rgbcolor='red',pointsize=100)}

\subsubsection{混合整数线性规划}
假如我们要求解下面的整数规划问题,要求目标函数$w_3$的最小值,约束条件为
\[
\begin{array}{lcr}
w_0+w_1+w_2-14w_3&==&0\\
w_1+2w_2-8w_3    &==&0\\
2w_2-3w_3         &==&0\\
w_0-w_1-w_2        &>=&0\\
w_3                  &>=&1
\end{array}
\]
其中$w_i,i=0,1,2,3$都是整数,则我们可以用以下代码求解这个问题
\begin{sageblock}
p=MixedIntegerLinearProgram(maximization=False,solver="GLPK")
w=p.new_variable(integer=True)

p.add_constraint(w[0]+w[1]+w[2]-14*w[3]==0)
p.add_constraint(w[1]+2*w[2]-8*w[3]==0)
p.add_constraint(2*w[2]-3*w[3]==0)
p.add_constraint(w[0]-w[1]-w[2]>=0)
p.add_constraint(w[3]>=1)
_=[p.set_min(w[i],None) for i in range(0.4)]
p.set_objective(w[3])

objective=p.solve()
variables=['w_'+str(i)+'='+str(v) for i,v in p.get_values(w).iteritems()]
\end{sageblock}

求解结果如下
\[\textrm{objective}=\sage{objective}\]
\[\textrm{variables}=\sage{variables}\]

